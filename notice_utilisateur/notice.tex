\documentclass{article}

\usepackage[utf8]{inputenc}
\usepackage[T1]{fontenc}
\usepackage[french]{babel}
\usepackage{geometry}
\usepackage{hyperref}
\usepackage{graphicx}

\title{Notice d'utilisation du module de Réalité Augmentée \textit{PFA}}
\author{
    BANIDE Christian \\
    \texttt{cbanide@bordeaux-inp.fr}
    \and
    DESBOIS-RENAUDIN Méo \\
    \texttt{mdesboisren@bordeaux-inp.fr}
    \and
    DESCOMPS Théo \\
    \texttt{tdescomps@bordeaux-inp.fr}
    \and
    DUCAMP Simon \\
    \texttt{siducamp@bordeaux-inp.fr}
    \and
    MORENO CARPIO Kenzo \\
    \texttt{kmcarpio@bordeaux-inp.fr}
    \and
    PIETTE Camille \\
    \texttt{camille.piette@bordeaux-inp.fr}
}
\date{\today}

\begin{document}

\maketitle

\tableofcontents

\section{Introduction}
Cette notice est à destination des développeurs de SpaceReset, ou tout autre utilisateur du module de réalité augmentée \textit{PFA}.
Elle détaille comment installer, prendre en main, et configurer le module. Elle présente aussi tous les contenus du module.

\section{Configurations}

Certains contenus du module de réalité augmentée sont modulables, et leur édition est laissée à la responsabilité de SpaceReset.
Ces contenus comprennent :
\begin{enumerate}
    \item Le marqueur faisant apparaître l'hologramme
    \item L'hologramme à afficher
    \item Le format du panel de l'hologramme
    \item Les informations du panel de l'hologramme
\end{enumerate}
Les sous-sections suivantes détaillerons la condiguration spécifique de chacun de ces contenus.

\subsection{Marqueur}

Le marqueur, soit l'objet détecté par le module d'où apparaîtra l'hologramme, s'installe dans la scène à la compilation.
Il se présente dans le cadre du PFA (et de AR Foundation en général) sous la forme d'une image.
Le PFA ne prévoit pas de changement de marqueur à l'exécution : \textbf{une fois l'application build avec un marqueur, il 
ne pourra pas être changé jusqu'au prochain build}. AR Foundation permet l'acceptation de plusieurs marqueurs en simultanée
par un système de \textit{Reference Libraries} (bibliothèques de références).

\vspace{5mm}

Afin d'installer un marqueur, importez votre image dans Unity, puis ajoutez-là dans une \textbf{Reference Image Library}.
Vous pouvez créer une Reference Image Library via : Clic-droit dans votre dépôt de projet, $\rightarrow$ Create 
$\rightarrow$ XR $\rightarrow$ Reference Image Library. 

\vspace{5mm}

Enfin, afin de charger la bibliothèque dans une scène RA, ajoutez votre Reference Image Library dans l'origine RA. 
Cliquez sur le GameObject "AR Session Origin" dans votre hiérarchie de scène, et dans son script "AR Tracked Image Manager"
glissez-déposez la bibliothèque dans le champ "Serialized Library".

\subsection{Hologramme}

L'hologramme, soit l'objet affiché en 3D, se décide via un Prefab, et peut donc se
changer à l'exécution. Un objet affiché en 3D dans l'environnement RA fonctionne en
terme d'affichage et de collisions comme un GameObject dans une scène 3D Unity classique.

\vspace{5mm}

Le Prefab définissant l'hologramme à afficher est stocké dans l'origine RA de la scène.
Cliquez sur le GameObject "AR Session Origin" dans votre hiérarchie de scène,
le champ "Tracked Image Prefab" dans son script "AR Tracked Image Manager" est l'hologramme.

\vspace{5mm}

Dans le cadre du PFA, le Prefab déjà chargé se nomme "SatellitePanel", et stocke à la fois l'hologramme
à afficher mais aussi son panel ; il dispose d'un script qui sert à l'instanciation de ces deux objets
\textbf{ce Prefab est fixé, ne tentez pas de le changer au risque de casser le fonctionnement attendu du module}.
Préférez la démarche suivante : que vous utilisiez une scène initiale différente de la scène RA ou non,
modifiez l'hologramme dans le champ "Hologram" du "DataInjector". Ce dernier s'occupera au chargement
de la scène RA d'injecter le bon hologramme dans le Prefab "SatellitePanel".

Cette modification du champ peut se faire à l'exécution (dans le cadre par exemple) de la récupération de
l'hologramme via un appel API.

\subsection{Format du panel}

Le panel est l'objet servant à décrire les informations de l'hologramme (Prefab "HologramPanel"). 
Il apparaît au touché de l'hologramme et disparaît à un touché dans le vide. 
Son format, soit l'ensemble de ses entrées, est fixé à l'éditeur.

\vspace{5mm}

Pour fixer le format du panel, ouvrez son Prefab. Allez dans le dossier des Prefabs "HoloEntries".
De là, vous pouvez glisser-déposer les entrées dans le "Body" du "HologramPanel" et vous verrez son
format se modifier en direct.

\textbf{Information cruciale :} la plupart des entrées suit un même format (taille verticale de 1),
mais certaines entrées font exceptions à la règle. Ces entrées exceptions ne doivent ni commencer ni finir
le panel (elles doivent être strictement à l'intérieur du body), sous risque de créer un format avec un bug
d'affichage.

\subsection{Informations du panel}

Le panel est l'objet servant à décrire les informations de l'hologramme (Prefab "HologramPanel"). 
Il apparaît au touché de l'hologramme et disparaît à un touché dans le vide. 
Son contenu, soit les informations inscrites dans chaque entrée, est changeable à l'exécution.

\vspace{5mm}

La démarche pour l'injection du contenu dans le panel est la même que pour l'injection de l'hologramme :
passer par un transmetteur de données inter-scènes (qui fonctionne aussi si présent sur une seule scène).

Modifiez le dictionnaire "panelContent" du Prefab "DataInjector" (possible via un script), 
ce dernier s'occupera au chargement de la scène RA d'injecter le bon contenu 
dans le Prefab "HologramPanel".


\end{document}
